\chapter{Review of Sets and Mathematical Notation}
In this chapter, we provide a brief review of fundamental mathematical notations that will be used throughout the coursework. \\
This involves set notation, set properties, as well as convenient mathematical notations that aid to simplify, abbreviate, as well as clarify the different conditions and clauses that we used to literally write out in mathematical statements.

\section{Set Theory}
This chapter describes the fundamental properties of mathematical objects called sets. \\
Sets are widely applied throughout and beyond the subfields of mathematics, whose origins and characteristics shared across those applications are described by the mathematical definition of sets.

\subsection{Fundamental Properties of Sets}
As you may have guessed, in set theory, we mainly discuss a mathematical object called "set".
\begin{ln-define}{Set}{}
    A set is a collection of objects. \\
    The mathematical objects that are members of a set are called elements. \\
    Sets are normally expressed in the format of:
    \[\{\text{element 1}, \text{element 2}, \dots, \text{element n}\}\]
\end{ln-define}
The symbol representing membership in a set works as follows:
\begin{ln-symbol}{Membership Symbol}{}
    If a mathematical object $x$ is a member of a set $A$, then we may mathematically express that $x \in A$. \\
    Otherwise, for a mathematical object $y$ that does not belong to $A$, $y \notin A$.
\end{ln-symbol}
There are several more properties to sets:
\begin{bindenum}
    \item The size of a set is defined as the number of elements in it. This property of set is called \textbf{cardinality}. The cardinality of a set $A$ can be mathematically denoted as $|A|$.
    \item The equality of a set is held when two sets have exactly same elements; order and repetition does not matter, albeit in computers, many implementations of sets already do not allow repetition. We call a set whose elements can be repeated a ``multiset''.
    \item Sets whose cardinality are 0, also known as empty sets, are denoted as $\{\}$ as well as $\emptyset$. \\
\end{bindenum}
To denote a set of mathematical objects that all belong to some other set but suffices some property, we may utilize a notation called \textbf{set-builder notation}:
\begin{ln-symbol}{Set-Builder Notation}{}
    A set $A$ whose members all follow expression $exp$, and the terms of $exp$ all fit some list of conditions can be written as:
    \[\{exp : \text{condition 1}, \dots, \text{condition n}\}\]
    \tcblower
    The colon above can be replaced by a vertical bar. Since we will be using vertical bars for other purposes in later chapters, a colon is perhaps the least abusive notation to play with. \\
    But, to conform the formatting of official lecture notes, let us continue with vertical bars. \\
    For example, the list of all rational numbers can be written as:
    \[\Q = \bigg\{ \frac{p}{q} \bigg| p,q \in \R,\ q \neq 0 \bigg\}\]
\end{ln-symbol}

\subsection{Relationships between Sets}
We often compare sets in terms of their sizes and elements they contain. Out of these standards for comparison, we can define the relationship between a set and another larger set that contains all elements of the former as follows:
\begin{ln-define}{Subset}{}
    If every element of a set $A$ is also in a set $B$, then $A$ is a subset of $B$. \\
    Mathematically, we write it as $A \subseteq B$. \\
    Or, stating the equivalent in an opposite direction, $B \supseteq A$, and this would state $B$ as a superset of $A$.
\end{ln-define}
And a stricter similar relationship follows:
\begin{ln-define}{Strict Subset}{}
    If $A \subseteq B$ but $A$ excludes at least an element of $B$, then we say that $A$ is a strict subset (proper subset) of $B$. \\
    Mathematically denoted, $A \subset B$.
\end{ln-define}
Utilizing the definitions above, we may form some observations:
\begin{bindenum}
    \item The empty set is a proper subset of any nonempty set.
    \item The empty set is a subset for any set, including itself's.
    \item While every set is not a proper subset of itself, every set is a subset of itself.
\end{bindenum}
While we compare sets based on their members, we can also "add", create sets based on the members of two sets. There are two ways of doing so, being \textbf{intersection} and \textbf{union}.
\begin{ln-define}{Intersection}{}
    The intersection of set $A$ and $B$, written as $A \cap B$, is the set containing all elements that are both in A and B. \\
    In set builder notation:
    \[A \cap B = \{x | x \in A \land x \in B\}\]
\end{ln-define}
\begin{ln-define}{Union}{}
    The union of set $A$ and $B$, written as $A \cup B$, is the set of all elements contained in either $A$ or $B$. \\
    In set builder notation, it pronounces very similarly with the notation for intersections:
    \[A \cup B = \{x | x \in A \lor x \in B\}\]
\end{ln-define}
Regarding the way empty sets work in the above arithmetic, for an arbitrary set $A$:
\begin{bindenum}
    \item $A \cap \emptyset = \emptyset$
    \item $A \cup \emptyset = A$
\end{bindenum}
At last, let us regard the notion of \textbf{``complements'', the difference between sets}.
\begin{ln-define}{Complement}{}
    Imagine the arithmetic of sets to work solely upon their members, and let the difference of sets be defined such that:
    \[A - B = A \backslash B = \{x \in A | x \notin B\}\]
    This set $A \ B$ is called the set difference between A and B, or alternatively, the relative complement of B in A.
\end{ln-define}
This also indicates that the ``subtraction arithmetic'' for sets is not commutative. Using empty sets as examples:
\begin{bindenum}
    \item $A\ \backslash\ \emptyset = A$
    \item $\emptyset\ \backslash\ A = \emptyset$
\end{bindenum}
And the last arithmetic is the \textbf{``multiplication of sets'': Cartesian Products}.
\begin{ln-define}{Cartesian Products}{}
    The Cartesian Product (cross product) of sets $A$ and $B$ is defined such that:
    \[A \times B = \{(a, b)\ |\ a \in \R, b \in \R\}\]
\end{ln-define}
And last but not least, we have mathematical operations that generate a set of sets (nested) based on other sets:
\begin{ln-define}{Power Set}{}
    The power set of $S$ can be written as:
    \[\wp (S) = \{P | P \subseteq S\}\]
    as one of its various denotations, the one on lecture note using a Weierstrass P. \\
    The power set of a set is the set of all of its possible subsets.
\end{ln-define}
In addition, if the cardinality of $S$ in the above definition is $|S| = k$, then we can state that $|\mathbb{P} (S)| = 2^k$. This will be an explored notion as we discuss a discrete math topic ``Counting'' in near future.

\section{Mathematical Notation}
Here we describe and explore mathematical notations that will abbreviate equations, statements, as well as facilitate us to view mathematical expressions in different perspectives. Mathematical notations introduced in this section will be very frequently employed in coming courseworks.

\subsection{Summation and Products}
The summation of some expression dependent on some other variable can be stated with a summation symbol, such that:
\begin{ln-symbol}{Summation}{}
    Let $f(x)$ be an expression based on some input $x$, then the following expressions are equivalent:
    \[f(m) + f(m + 1) + \dots + f(m + n) = \sum_{k = 1}^{n} {f(m + k)}\]
\end{ln-symbol}
While the summation expression is highly applicable to numerous theorems, there is another notation similar in nature to the summation expression-- the product expression:
\begin{ln-symbol}{Product}{}
    Let $f(x)$ be an expression based on some input $x$, then the following expressions are equivalent:
    \[f(m) \times f(m + 1) \times \dots \times f(m + n) = \prod_{k = 1}^{n} {f(m + k)}\]
\end{ln-symbol}

\subsection{Quantifiers}
Quantifiers are symbols that help us target a range of elements from some set when we need to write a conditional statement for it. There are two specific quantifiers we will use here.
\begin{ln-symbol}{Universal Quantifier}{}
    The universal quantifier $\forall$, pronounced as "for all", targets all elements from a set. \\
    For example, the phrase "$\forall x \in A$" is equivalent to the phrase "for all objects $x$ that belongs to the set $A$". This can serve as the first clause of a statement, while the second clause would refer a statement for the elements targeted by the quantifier.
\end{ln-symbol}
Meanwhile, we have a different quantifier that selects a different range from the universal quantifier:
\begin{ln-symbol}{Existential Quantifier}{}
    The existential quantifier $\exists$, pronounced as "there exists (at least one)", states that there should exist at least one element from some set that fits a condition stated in the second clause. The syntax, therefore, is as follows:
    \[ \exists x \in A : \text{statement}\]
    This means: there exists at least an object $x$ belonging to the set $A$ such that the "statement" holds. There are some more variants to this overall syntax, such as replacing $\in$ with $\notin$... etc.
\end{ln-symbol}
There also exists some variations to the existential quantifier:
\begin{bindenum}
    \item There exists at least one: $\exists$
    \item There exists only one: $\exists!$
    \item There does not exist: $\nexists$
\end{bindenum}

\section{Personal Regard on this Topic}
Personal log for mathematical topics, because why not! Just don't treat this section as educational content. \\
Sets are very useful objects for mathematical generalizations, which we'd do a TON in discrete mathematics, hence its practicality. There will, however, come times where statements are hard to write out. \\
In that case, don't bother with making the statements multi-part! There's a clear trend in homework questions and solutions for that. \\

Also, my reader didn't grade one of my homework questions and I missed my regrade deadline :v
