\chapter{Geometric and Poisson Distributions}

\section{Geometric Distribution}
The geometric distribution frequently occurs in situations where we consider ``how long we have to wait'' for something to occur, such as ``how many run until error'', or ``how many rolls until SSR''.

\subsection{Definition}
The \textbf{geometric distribution} is then mathematically defined as below:
\begin{ln-define}{Geometric Distribution}{}
    A random variable $X$ such that:
    \[\P[X = i] = {(1 - p)}^{i - 1} p, i \in \Z^+\]
    is said to have a geometric distribution with parameter $p$, mathematically expressable as:
    \[X \sim \text{Geometric}(p)\]
\end{ln-define}
Does such distribution sum to $1$? Allow us to verify:
\begin{align*}
    \sum_{i = 1}^{\infty} \P[X = i]
    &= \sum_{i = 1}^{\infty} {(1 - p)}^{i - 1} p \\
    &= p \sum_{i = 1}^{\infty} {(1 - p)}^{i - 1} \\
    &= p \lim_{n \rightarrow \infty} \frac{1 - {(1 - p)}^{n}}{1 - (1 - p)} \\
    &= p \frac{1}{1 - 1 + p} = 1
\end{align*}

The distribution of $X$ would thus be some curve that decreases by a factor of $(1 - p)$ at each step; in other words, an exponential decay. \\
From this fact, we would know that the height of curve is dependent on $p$, and the speed at which the curve decrease depends on $(1 - p)$.

\subsection{Mean and Variance of a Geometric RV}
Let us first discuss the expectation of a geometric random variable. \\
If directly computed, we would obtain some result like what follows:
\[\E[X] = \sum_{i = 1}^\infty i\P[X = i] = p\sum_{i = 1}^\infty i{(1 - p)}^{i - 1}\]
However, there is a way to simplify this above complication:
\begin{ln-theorem}{Tail Sum Formula}{}
    \textbf{Theorem}:
    \[\E[X] = \sum_{i = 1}^\infty \P[X \geq i]\]
    \tcblower
    \textbf{Proof}:
    \begin{align*}
        \E[X] &= \sum_{i = 1}^\infty i\P[X = i] \\
        &= p\sum_{i = 1}^\infty i{(1 - p)}^{i - 1} \\
        &= 0 p_0 + 1 p_1 + 2 p_2 + \cdots \\
        &= (p_1 + p_2 + \cdots) + (p_2 + \cdots) + \cdots \\
        &= \P[X \geq 1] + \P[X \geq 2] + \cdots
    \end{align*}
\end{ln-theorem}
What is the significance of this?
\begin{ln-theorem}{The Expectation of a Geometric RV}{}
    \begin{align*}
        \P[X \geq i] &= (1 - p){(1 - p)}^{i - 1} + p{(1 - p)}^{i - 1} \\
        &= {(1 - p)}^{i - 1} \\
        \E[X] &= \sum_{i = 1}^\infty \P[X \geq i] \\
        &= \sum_{i = 1}^\infty {(1 - p)}^{i - 1}
        = \frac{1}{1 - (1 - p)} = \frac{1}{p}
    \end{align*}
\end{ln-theorem}
Upon that, let's also discuss and derive such identity:
\begin{ln-theorem}{Expectation of X(X - 1)}{}
    \textbf{Theorem}:
    \[\E[X(X - 1)] = \frac{2(1 - p)}{p^2}\]
    \tcblower
    \textbf{Proof}:
    Preliminarily,
    \begin{align*}
        \E[X(X - 1)]
        &= \sum_{i = 1}^\infty i(i - 1) \times \P[X = i] \\
        &= \sum_{i = 1}^\infty i(i - 1) \times p{(1 - p)}^{i - 1} \\
        &= p(1 - p) \sum_{i = 1}^\infty i(i - 1) {(1 - p)}^{i - 2}
    \end{align*}
    With some calculus and an identity we just learned,
    \begin{align*}
        \sum_{i = 1}^\infty {(1 - p)}^i &= \frac{1}{p} \\
        \frac{d}{di} &= \sum_{i = 1}^\infty -i{(1 - p)}^{i - 1} \\
        &= -\frac{1}{p^2} \\
        \frac{d^2}{di^2} &= \sum_{i = 1}^\infty i(i - 1) {(1 - p)}^{i - 2} \\
        &= \frac{2}{p^3}
    \end{align*}
    Using the result above:
    \begin{align*}
        \E[X(X - 1)]
        &= p(1 - p) \sum_{i = 1}^\infty i(i - 1) {(1 - p)}^{i - 2} \\
        &= p(1 - p) \frac{2}{p^3} \\
        &= \frac{2(1 - p)}{p^2}
    \end{align*}
\end{ln-theorem}

Using the linearity of expectation, we can elegantly calculate the variance of a geometric distribution:
\begin{ln-theorem}{The Variance of a Geometric RV}{}
    \textbf{Theorem}:
    \[Var(X) = \frac{1 - p}{p^2}\]
    \tcblower
    \textbf{Proof}:
    \begin{align*}
        Var(X) &= \E[X^2] - {(\E[X])}^2 \\
        &= \E[X^2 - X] + \E[X] - {(\E[X])}^2 \\
        &= \frac{2(1 - p)}{p^2} + \frac{1}{p} - \frac{1}{p^2} \\
        &= \frac{2 - 2p + p - 1}{p^2} = \frac{1 - p}{p^2}
    \end{align*}
\end{ln-theorem}

\subsection{Memoryless}
What? Geometric Distributions can have Alzeihmers? This is the first time I have ever seen someone who's not Alzeihmerus be memoryless! \\
What? \\
\begin{ln-define}{Memoryless}{}
    In memoryless distributions, prior trials do not affect the number of additional trials needed. Mathematically:
    \[\P[X > n + m | X > n] = \P[X > n]\]
    For example, say we have already tossed the coin $m$ times, what is the probability that we need more than $n$ additional tosses before getting our first head?
    \begin{align*}
        \P[X > n + m | X > m]
        &= \frac{\P[X > n + m]}{\P[X > m]} \\
        &= \frac{{(1 - p)}^{n + m}}{{(1 - p)}^m} \\
        &= {(1 - p)}^n = \P[X > n]
    \end{align*}
\end{ln-define}
In this sense, ``memorylessness'' refers to how the distribution, when calculating a tail, will not remember the given condition ($X > m$) since it doesn't contribute to the end result of a tail even given that condition.

\subsection{Example (Application)}
What other greater application can CS70 be made on other than probabilistic problems, and more specifically, gacha games!? \\
Let us look at the ``Coupon Collector Problem'':
\begin{quote}
    Let us attempt to collect a set of $n$ different cards. \\
    We can get one card by performing one summon, each of the $n$ cards equivalently likely to appear from that summon. \\
    How many summons do we have to do until we have collected at least one copy of every card?
\end{quote}
Let $S_n$ denote the number of summoning we need to perform to collect all $n$ cards, then the distribution of $S_n$ is difficult to compute directly. But, its expected value can be calculated via the gift of linearity. \\
Let $X_i$ be a random variable rerpesenting the number of boxes needed while trying to get the $i^{th}$ new card, starting immediately after the most recent new card's obtain. \\
Then,
\[S_n = X_1 + X_2 + \cdots + X_n\]
Therefore,
\[\E[S_n] = \E[X_1 + X_2 + \cdots + X_n] = \E[X_1] + \E[X_2] + \cdots + \E[X_n]\]

The expected value of $X_1$ would be $1$, since any first card is a new card. \\
Now, how about $X_i$? \\
Each time we perform a summoning, we will get an new card with probability $\frac{n - i}{n}$. We will continue performing summoning, which can be characterized as a coin flip of heads probability $p = \frac{n - i}{n}$ for simplicity. This makes $X_i$ the number of tosses until first head. \\
There, 
\[X_i \sim \text{Geometric}(\frac{n - i}{n})\]
And consequentially,
\[\E[X_i] = \frac{n}{n - i}\]

Coming back to the original simplification of $\E[S_n]$, then:
\[
    \E[S_n] = \sum_{i = 1}^n \frac{n}{n - i} = n\sum_{i = 1}^n \frac{1}{i}
\]
This can be well approximated as
\[
    n\sum_{i = 1}^n \frac{1}{i} \simeq n(\ln{n} + \gamma_E)
\]
Where $\gamma_E=0.5772\dots$ is known as Euler's constant.

\section{Poisson Distribution}

\subsection{Definition}
To start off this discussion, let us first define its subject:
\begin{ln-define}{Poisson Distribution}{}
    A random variable X for which:
    \[\P[X = i] = \frac{\lambda^i}{i!} e^{-\lambda}, i \in \Z^+\]
    is said to have the Poisson distribution with paramerter $\lambda$. \\
    We may mathematically abbreviate such as:
    \[X \sim Poisson(\lambda)\]
\end{ln-define}
Let us, once again, verify the definition of distribution by testing whether it sums up to $1$:
\begin{align*}
    \sum_{i = 0}^\infty \frac{\lambda^i}{i!} e^{-\lambda}
    &= e^{-\lambda} \sum_{i = 0}^\infty \frac{\lambda^i}{i!} \\
    e^{-x} &= 1 + x + \frac{x^2}{2!} + \cdots \\
    \sum_{i = 0}^\infty \frac{\lambda^i}{i!} e^{-\lambda}
    &= e^{-\lambda} e^{\lambda} = 1
\end{align*}

\subsection{Properties of Poisson RV}
For both expectation and variance, we can discover a very elegant identity. \\
\begin{ln-theorem}{Expectation of Poisson RV}{}
    \begin{align*}
        \E[X]
        &= \sum_{i = 0}^\infty i \P[X = i] \\
        &= \sum_{i = 0}^\infty i \frac{\lambda^i}{i!} e^{-\lambda} \\
        &= e^{-\lambda} \sum_{i = 1}^\infty \frac{\lambda^i}{(i - 1)!} \\
        &= e^{-\lambda} \lambda \sum_{i = 1}^\infty \frac{lambda^{i - 1}}{(i - 1)!} \\
        &= e^{-\lambda} \lambda e^{\lambda} = \lambda
    \end{align*}
\end{ln-theorem}
And as for variance:
\begin{ln-theorem}{Variance of Poisson RV}{}
    \begin{align*}
        \E[X(X - 1)]
        &= \sum_{i = 0}^\infty i(i - 1) \P[X = i] \\
        &= \sum_{i = 0}^\infty i(i - 1) \frac{\lambda^i}{i!} e^{-\lambda} \\
        &= \lambda^2 e^{-\lambda} \sum_{i = 2}^\infty \frac{lambda^{i - 2}}{(i - 2)!} \\
        &= \lambda^2 e^{-\lambda} e^{\lambda} = \lambda^2 \\
        Var(X) &= \E[X^2] - {(\E[X])}^2 \\
        &= \E[X(X - 1)] + \E[X] - {(\E[X])}^2 \\
        &= \lambda^2 + \lambda - \lambda^2 = \lambda
    \end{align*}
\end{ln-theorem}

\begin{ln-theorem}{Superposition of Poisson Distribution}{}
    \textbf{Theorem}: Let $X \sim Poisson(\lambda)$ and $Y \sim Poisson(\mu)$ \underline{be independent of each other}, then $X + Y \sim Poisson(\lambda + \mu)$.
    \tcblower
    \textbf{Proof}:
    \begin{align*}
        \P[X + Y = k]
        &= \sum_{j = 0}^k \P[X = j, Y = k - j] \\
        &= \sum_{j = 0}^k \P[X = j] \P[Y = k - j] \\
        &= \sum_{j = 0}^k \frac{\lambda^j}{j!} e^{-\lambda} \frac{\mu^{k - j}}{(k - j)!} e^{-\mu} \\
        &= e^{-(\lambda + \mu)} \sum_{j = 0}^k \frac{\lambda^j \mu^{k - j}}{j!(k - j)!} \\
        &= e^{-(\lambda + \mu)} \frac{{(\lambda + \mu)}^k}{k!}
    \end{align*}
\end{ln-theorem}

\begin{ln-theorem}{Poisson as a Limit of Binomial}{}
    \textbf{Theorem}: Let $X \sim Binomial(n, \frac{\lambda}{n})$ where $\lambda > 0$ is constant, then
    \[\forall i \in \N, \lim_{n \rightarrow \infty}\P[X = i] = \frac{\lambda^i}{i!}e^{-\lambda}\]
    \tcblower
    \textbf{Proof}:
    First, let's do some algebraic simplification.
    \begin{align*}
        \P[X = i]
        &= \binom{n}{i} {(\frac{\lambda}{n})}^i {(1 - \frac{\lambda}{n})}^{n - i} \\
        &= \frac{\lambda^i}{i!} \cdot \frac{n!}{(n - i)! n^i} \cdot {(1 - \frac{\lambda}{n})}^{n} \cdot {(1 - \frac{\lambda}{n})}^{-i}
    \end{align*}
    Here,
    \[\frac{n!}{(n - i)! n^i} = \frac{n (n - 1) (n - 2) \cdots (n - i + 1)}{n^i}\]
    This value approaches $1$ as $n$ approaches $\infty$. \\
    Meanwhile, from calculus, we may determine that:
    \[\lim_{n \rightarrow \infty} {(1 - \frac{\lambda}{n})}^n = e^{-\lambda}\]
    And last but not least, since $i$ is a fixed variable,
    \[\lim_{n \rightarrow \infty} {(1 - \frac{\lambda}{n})}^{-i} = 1\]
    Therefore, multiplying these limits together, we obtain that:
    \[\lim_{n \rightarrow \infty} \P[X = i] = \frac{\lambda^i}{i!}e^{-\lambda}\]
\end{ln-theorem}
